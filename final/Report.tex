\documentclass[a4paper, 12pt]{report}

\usepackage[style=numeric]{biblatex}
\usepackage[utf8]{inputenc}
\usepackage{geometry}
\usepackage{graphicx}
\usepackage{amsmath}
\usepackage{hyperref}

\addbibresource{References.bib}
\graphicspath{ {images/} }

\title{
	{Fail-Safe Cloud Tournament Engine with Error Detection and Error Recovery} \\
	{\includegraphics[scale=1.4]{logo.png}}
}
\author{
	{Kyle Chapman (20703236)} \\
	{\emph{Supervisor: Dr. Cornelia P. Inggs}} \\
	{\emph{Cosupervisor: Mr. Andrew J. Collett}}
}
\date{9 November 2020}

\begin{document}

\maketitle
\tableofcontents

\chapter{Introduction}

The Fail-Safe Cloud Tournament Engine (FSCTE) is an extension of the Cloud
Tournament Engine (CTE) developed by Reece Murray in 2018, which is an extension
of the Tournament Engine (TE) built on the Ingenious Framework\footnote{\url{https://bitbucket.org/skroon/ingenious-framework/src/master/}}
(IF). The goal for the CTE is to allow people to play a turn-based game, provided
it is supported in the IF, against each other in a match, where many matches
make up a tournament. Multiple tournaments can be run to determine how good
each player is. Tournaments can separated into two categories: private and public. \\

Private tournaments are tournaments that only the administrators can add players
to, and they are mainly used to test some players against other players. Public
tournaments are tournaments that any user can add their own players to and
administrators can also add players to. These tournaments are used to put players
against each other to rank each player based on how many wins they get against
other players. \\

It was found that there were some major limitations to the CTE, such as minimal
error reporting, lack of error recovery and an unmaintainable code base to name
a few. As a result, there were situations that arose in which a player failure
resulted in a failure of the entire tournament. This prompted the need for a
version that was fail-safe and included error reporting and error recovery, as
far as possible. The FSCTE aims to address these limitations and improve the
overall system stability by being fail-safe, which means that if any error were
to occur, the response to the error would cause as little harm to the overall
system stability as possible.

\section{Scope}

The system used to manage the FSCTE is made up of multiple Docker\footnote{\url{https://docs.docker.com/engine/docker-overview/}}
containers, which run each section of the FSCTE in their own separate environment.
If some error occurs in a section of the system, it will not affect the other
sections, since they are in separate environments. \\

The FSCTE uses the following three main sections that are run in their own
Docker containers:

\begin{itemize}
	\item \textbf{Web User Interface}: Allows a user to interact directly with
	the FSCTE.
	\item \textbf{Database}: Stores all the information relevant to the FSCTE.
	\item \textbf{Database watcher (Golang)}: Watches the database for any changes,
	such as tournaments being started, and sends messages to the front-end
	accordingly.
	\item \textbf{REST API}: Facilitates file upload to, and download from, the
	web user interface, as well as CAS authentication for logging in with
	Stellenbosch University credentials.
\end{itemize}

The layout of the entire FSCTE system is further discussed in chapter
\ref{chap:design}, \emph{design and implementation}.

\section{Document Outline}

This report will discuss the overview of the system in chapter \ref{chap:overview},
the implementation and design of the system in chapter \ref{chap:design}, then
onto the testing in chapter \ref{chap:testing} and discussion of the future of
the project in chapter \ref{chap:future}. Finally, the conclusion will be
presented in chapter \ref{chap:conclusion}.

\chapter{Overview}
\label{chap:overview}

This chapter will give an overview of the requirements for the FSCTE system and
how the requirements have been met. \\

The CTE used in $3^{\text{rd}}$ year at Stellenbosch
University is designed to manage many Othello \cite{othello} tournaments,
involving many players, concurrently. A user is able to view public tournaments,
players, referees, schedulers, etc. and view relevant statistics for
ongoing and completed tournaments. The admin, usually the lecturer, is
able to add tournaments, schedulers and rankers and the students are only allowed
to upload their players and add them to public tournaments.

\section{Separate enviroments}

Each section of the FSCTE needs to be in their own separate environment, so that
if some error occurs, it impacts only that specific section and not the entire
FSCTE. Docker was chosen for this purpose, since it is relatively easy to spin
up a section in its own environment and redeploy that section if some error
occurs. \\

The docker container is created by writing a \texttt{Dockerfile} that
specifies which base environment you want to use (e.g. \texttt{ubuntu},
\texttt{node}, \texttt{neo4j} etc.) and the command you want to run in the
container. The container exits when the command finishes, so in order to keep
the container running forever, infinite loops are useful. This means that the
command will only finish when the user forcefully stops the container. \\

Docker-compose\footnote{\url{https://docs.docker.com/compose/}} is used to spin
up all of the docker containers used by the FSCTE,
using their respective \texttt{Dockerfile}s, and allows for more control of how
the container should be set up. Local directories can be mounted into the docker
containers so that any files that are generated inside the container can be
accessed outside of the container, on the host machine. This is useful for
accessing the log files after a match has completed, so that it can be seen if
the match was a success or not.

\section{Web User Interface}

The previous web user interface implemented in the CTE was not very user-friendly,
which resulted in a lot of confusion for the students who used the CTE.

\chapter{Design and Implementation}
\label{chap:design}

A scheduler is used to schedule matches between two players in a round-robin
\mbox{fashion}, where every player plays against every other player. After every player
has played against every other player, the overall win-loss ratio can be calculated
that can be used to \mbox{distinguish} between good, average and bad players. The
win-loss ratio and the Elo rating \cite{elo} can be used in conjunction. \\

A ranker is used to distinguish between the good and bad players by comparing
the Elo of the two players, where every player starts on the same amount of Elo.
For every win that a player has, they gain a certain amount of Elo and for every
loss that a player has, they lost a certain amount of Elo. The amount gained or
lost decreases for every match that the player plays, meaning that eventually
the Elo rating will be a true reflection of the skill level of the player. \\

A referee is used to monitor the moves that the two players make in a match and
check if any player makes an invalid move or times out. One player sends a
move to the referee, which then checks if the move is valid and only sends
the move to the other player if the move made is valid. If the move made is
invalid, or the player times out, the match will be forfeit and the last player
to send a valid move becomes the winner.

\chapter{Testing}
\label{chap:testing}

\chapter{Future Work}
\label{chap:future}

This chapter covers the future work that could be implemented in the Fail-Safe
Cloud Tournament Engine (FSCTE) to extend the current functionality. The
conclusion will be provided in section \ref{chap:conclusion}.

\section{Kubernetes}

Kubernetes\footnote{\url{https://kubernetes.io/}} can be used to manage the docker
containers, so that if a container crashes, another instance of that container
can be redeployed without hesitation. Due to time constraints, it was not feasible
to add Kubernetes to the current FSCTE.

\section{Other Game Types}

In the current FSCTE implementation, only the \texttt{Othello} game is supported.
Supporting other game types will allow the FSCTE to be more abstract and versatile.
Due to time constraints, it was not feasible to attempt to support other types
of games.

\chapter{Conclusion}
\label{chap:conclusion}

The main requirements of the FSCTE that needed to be satisfied were:
\begin{itemize}
	\item have effective error handling
	\item have effective debugging facilities
	\item scale with tournaments and players
	\item have a logical, clean and maintainable code base
	\item have a more user-friendly web user interface
\end{itemize}
All of the above requirements were satisfied. The error handling requirement
was satisfied by making sure that at any point where an error could occur, it
was decided if the error was serious or not. If the error was serious, then an
error message would be displayed and the component that caused the error would
be stopped. If the error was not serious, then an error message would be displayed,
but nothing would happen to the component that caused the error. \\

The effective debugging facilities requirement was satisfied by logging all
information that happens inside each docker container, so that if some error
occurs, the user can look in the docker logs to see at which stage the error
occurred. This allows the user to know exactly what was going on before the
error occurred so that the user can pinpoint the issue and fix the problem. \\

The tournaments and player scaling requirement was satisfied by writing the
code base in such a way that it performs very well for any number of tournaments
and players. Golang was mainly used to allow scaling to be possible through the
use of concurrent threads running different tasks at the same time. \\

The maintainable code base requirement was satisfied by providing extensive
documentation throughout the code base, so that anyone who reads the code base
can understand what is going on without any issue. \\

The user-friendly web user interface requirement was satisfied by ensuring that
the user can navigate to any page or get any information they want in a maximum
of two mouse clicks. A lot of time was spent making sure that the web user
interface looked clean and visually appealing, as well as easy to use. \\

Working on the FSCTE for the past year has been a great challenge, but also a
great reward. Initially, it was tough to understand what was going on in the CTE
because of the lack of documentation and confusing way that some sections were
implemented. After understanding the previous work, the toughest challenge was
getting the players to play a match and communicate their results to the web
user interface. However, it was a fun and informative experience.

\printbibliography

\end{document}
