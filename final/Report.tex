\documentclass[a4paper, 12pt]{report}

\usepackage[style=numeric]{biblatex}
\usepackage[margin=2.5cm]{geometry}
\usepackage[utf8]{inputenc}
\usepackage{graphicx}
\usepackage{amsmath}
\usepackage{hyperref}

\addbibresource{References.bib}
\graphicspath{ {images/} }

\title{
	{Fail-Safe Cloud Tournament Engine with Error Detection and Error Recovery} \\
	{\includegraphics[scale=1.4]{logo.png}}
}
\author{
	{Kyle Chapman (20703236)} \\
	{Supervisor: Dr. Cornelia P. Inggs} \\
	{Cosupervisor: Mr. Andrew J. Collett}
}
\date{November 2020}

\begin{document}

\maketitle
\tableofcontents

\chapter{Introduction}

The Cloud Tournament Engine (CTE) used in $3^{\text{rd}}$ year at Stellenbosch
University is \mbox{designed} to manage many Othello \cite{othello} tournaments,
involving many players, concurrently. A user is able to view public tournaments,
players, referees, schedulers, etc. and view \mbox{relevant} statistics for
ongoing and completed tournaments. The admin, usually the \mbox{lecturer}, is
able to add tournaments, schedulers and rankers.

\vspace{0.8em}
A scheduler is used to schedule matches between two players in a round-robin
\mbox{fashion}, where every player plays against every other player. After every player
has played against every other player, the overall win-loss ratio can be calculated
that can be used to \mbox{distinguish} between good, average and bad players. The
win-loss ratio and the Elo rating \cite{elo} can be used in conjunction.

\vspace{0.8em}
A ranker is used to distinguish between the good and bad players by comparing
the Elo of the two players, where every player starts on the same amount of Elo.
For every win that a player has, they gain a certain amount of Elo and for every
loss that a player has, they lost a certain amount of Elo. The amount gained or
lost decreases for every match that the player plays, meaning that eventually
the Elo rating will be a true reflection of the skill level of the player.

\vspace{0.8em}
A referee is used to monitor the moves that the two players make in a match and
check if any player makes an invalid move or times out. One player sends a
move to the referee, which then checks if the move is valid and only sends
the move to the other player if the move made is valid. If the move made is
invalid, or the player times out, the match will be forfeit and the last player
to send a valid move becomes the winner.

\vspace{0.8em}
The current CTE was used in 2019, but it was found to have some major limitations.
The web interface that is used is not user-friendly, there is minimal to no
error reporting or error recovery, the code base is umaintainable and it
relies on older technology. In many scenerios, a player failure results in
a failed tournament when running the CTE.

\chapter{System Overview}

\chapter{Design and Implementation}

\chapter{Testing}

\chapter{Future Work}

\chapter{Conclusion}

\printbibliography

\end{document}
