\documentclass[a4paper, 12pt]{report}

\usepackage[style=numeric]{biblatex}
\usepackage[margin=2.5cm]{geometry}
\usepackage[utf8]{inputenc}
\usepackage{graphicx}
\usepackage{amsmath}
\usepackage{hyperref}

\addbibresource{References.bib}
\graphicspath{ {images/} }

\title{
	{Fail-Safe Cloud Tournament Engine with Error Detection and Error Recovery} \\
	{\includegraphics[scale=1.4]{logo.png}}
}
\author{
	{Kyle Chapman (20703236)} \\
	{Supervisor: Dr. Cornelia P. Inggs} \\
	{Cosupervisor: Mr. Andrew J. Collett}
}
\date{November 2020}

\begin{document}

\maketitle
\tableofcontents

\chapter{Introduction}

The Fail-Safe Cloud Tournament Engine (FSCTE) is an extension of the Cloud
Tournament Engine (CTE) developed by Reece Murray in 2018, which was an extension
of the Tournament Engine (TE) built on the Ingenious Framework\footnote{\url{https://bitbucket.org/skroon/ingenious-framework/src/master/}}
(IF). The goal for the CTE was to allow people to play a turn-based game, provided
it was supported in the IF, against each other in a match, where many matches
made up a tournament. Multiple tournaments were then run to determine how good
each player was. \\

It was found that there were some major limitations to the CTE, such as minimal
error reporting, lack of error recovery and an unmaintainable code base to name
a few. As a result, there were situations that arose in which a player failure
resulted in a failure of the entire tournament. This prompted the need for a
version that was fail-safe and included error reporting and error recovery, as
far as possible. The FSCTE aims to address these limitations and improve the
overall system stability by being fail-safe, which means that if any error were
to occur, the response to the error would cause as little harm to the overall
system stability as possible.

\section{Scope}

The system used to manage the FSCTE is made up of multiple Docker\footnote{\url{https://docs.docker.com/engine/docker-overview/}}
containers, which run each section of the FSCTE in their own separate environment.
If some error occurs in a section of the system, it will not affect the other
sections, since they are in separate environments. \\

The FSCTE uses the following three main sections that are run in their own
Docker containers:

\begin{itemize}
	\item \textbf{Web User Interface}: Allows a user to interact with the FSCTE.
	\item \textbf{Database}: Used to store all the information relevant to the FSCTE.
	\item \textbf{REST API}: Facilitates file upload to, and download from, the
	web user interface.
\end{itemize}

\chapter{System Overview}

\section{Ingenious Framework (IF)}

The CTE used in $3^{\text{rd}}$ year at Stellenbosch
University is \mbox{designed} to manage many Othello \cite{othello} tournaments,
involving many players, concurrently. A user is able to view public tournaments,
players, referees, schedulers, etc. and view \mbox{relevant} statistics for
ongoing and completed tournaments. The admin, usually the \mbox{lecturer}, is
able to add tournaments, schedulers and rankers. \\

A scheduler is used to schedule matches between two players in a round-robin
\mbox{fashion}, where every player plays against every other player. After every player
has played against every other player, the overall win-loss ratio can be calculated
that can be used to \mbox{distinguish} between good, average and bad players. The
win-loss ratio and the Elo rating \cite{elo} can be used in conjunction. \\

A ranker is used to distinguish between the good and bad players by comparing
the Elo of the two players, where every player starts on the same amount of Elo.
For every win that a player has, they gain a certain amount of Elo and for every
loss that a player has, they lost a certain amount of Elo. The amount gained or
lost decreases for every match that the player plays, meaning that eventually
the Elo rating will be a true reflection of the skill level of the player. \\

A referee is used to monitor the moves that the two players make in a match and
check if any player makes an invalid move or times out. One player sends a
move to the referee, which then checks if the move is valid and only sends
the move to the other player if the move made is valid. If the move made is
invalid, or the player times out, the match will be forfeit and the last player
to send a valid move becomes the winner.

\chapter{Design and Implementation}

\chapter{Testing}

\chapter{Future Work}

\chapter{Conclusion}

\printbibliography

\end{document}
